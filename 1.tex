\section{color column}

:set colorcolumn=80

forma acortada:

:set cc=80

desactivar color column:

:set cc=

\section{pegar texto desde fuera de vim}

"+p

Esa combinación de letras estando en el modo normal, tambien hay otra forma
estando con el teclado númerico desactivado. la tecla bloq num. luego copiar
el texto desde fuere de vim luego estando dentro de vim en el modo normal

:set paste

luego presionar i para ingresar en el modo insertar para luego presionar
la tecla Shift +  insert del teclado númerico.

\section{.vimrc}

Podemos ver que el archivo .vimrc antes de la version 9 tenia que ser
escrito con un guion bajo en la version de vim para windows llamado gvim.

\begin{lstlisting}
":syntax on

:set cc=80

:colorscheme habamax

:set number
:set encoding=utf-8

:set noundofile
:set nobackup
:set noswapfile

:set ts=4 sw=4 sts=4
:set autoindent
:set backspace=2
:set backspace=indent,eol,start

inoremap " ""<left>
inoremap ( ()<left>
inoremap [ []<left>
inoremap { {}<left>

"let g:user_emmet_mode='n' "Only enable normal mode functions.
"let g:user_emmet_leader_key=','

inoremap <c-b> <Esc>:Lex<cr>:vertical resize 30<cr>
nnoremap <c-b> <Esc>:Lex<cr>:vertical resize 30<cr>

nnoremap <c-s> :w<CR>
nnoremap . :

call plug#begin()
Plug 'OmniSharp/omnisharp-vim'
call plug#end()

let g:OmniSharp_server_path = 'C:\omnisharp-vim\omnisharp-roslyn\OmniSharp.exe'

filetype indent plugin on
syntax enable

let g:OmniSharp_server_use_net6 = 1

let g:OmniSharp_highlighting = 3
\end{lstlisting}

Este archivo esta dentro del directorio de instalación c:/Program Files/vim
pero los plugins estan en el directorio home, es decir que users/usuario/..
